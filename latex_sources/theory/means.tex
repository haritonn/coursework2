\subsection{Основные понятия компьютерного зрения}
Компьютерное зрение "--- область искуственного интеллекта, которая занимается разработкой и оптимизацией технологий для обработки визуальной информации из изображений и видео. Основная суть комьютерного зрения "--- получение более структурированной информации, пригодной для дальнейшего использования в приложениях \cite{cv_intro}.

Среди процессов решения задачи компьютерного зрения можно выделить:
\begin{itemize}
    \item Сбор визуальных данных "--- получение исходного набора данных для дальнейшего обучения модели;
    \item Предобработка данных "--- подготовка данных для модели: изменение размера, нормализация яркости, обрезка ненужной информации;
    \item Интерпретация "--- применение алгоритмов классического машинного обучения и глубокого обучения для решения определённой задачи. Например, решение задачи классификации или сегментации. 
\end{itemize}

Глубокое обучение является одним из ключевых направлений машинного обучения. Решения из глубокого обучения предполагают использование многослойных искусственных нейронных сетей для автоматического извлечения признаков для дальнейшей классификации или распознавания. 

Генеративные модели представляют собой класс алгоритмов, конечной целью которых является аппроксимация распределения исходного набора данных для дальнейшего синтеза данных. В контексте компьютерного зрения, генеративные модели применяются для генерации новых, фотореалистичных изображений. Именно эти модели применяются для решения задачи виртуальной примерочной, поскольку необходимо сгенерировать новое изображение исходной модели с другой одеждой. К генеративным моделям относят порождающие состязательные сети, диффузионные модели, а также вариацонные автокодировщики \cite{generative_plus_gan}.