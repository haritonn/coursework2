\section{Вычисление \texttt{LPIPS}}\label{app:lpips}
Требуемое содержимое корня:
\begin{verbatim}
    computing_metrics/
    |--GAN/
    |--LADI/
    |--PROMPT/
    |--image/
    |--compute_lpips.py
\end{verbatim}
Содержимое файла \texttt{compute\_lpips.py}:
\begin{minted}[fontsize=\small, breaklines=true, style=bw, linenos]{python}
import lpips
import torch
import matplotlib.pyplot as plt
import numpy as np
from PIL import Image
import os

device = 'cuda' if torch.cuda.is_available() else 'cpu'

# Инициализация LPIPS
loss_fn = lpips.LPIPS(net='alex').to(device)

def load_image(path):
    img = Image.open(path).convert('RGB')
    img = img.resize((256, 256))  
    img_tensor = torch.from_numpy(np.array(img)).permute(2,0,1).float() / 255.0
    img_tensor = img_tensor.unsqueeze(0) * 2 - 1 
    return img_tensor

def calculate_lpips(source_folder, result_folder, source_files, result_files):
    scores = []
    
    for src, res in zip(source_files, result_files):
        if src.split('.')[0] == res.split('.')[0]:
            src_path = os.path.join(source_folder, src)
            res_path = os.path.join(result_folder, res)

            img1 = load_image(src_path).to(device)
            img2 = load_image(res_path).to(device)

            with torch.no_grad():
                dist = loss_fn(img1, img2)
            scores.append(dist.item())
    return scores

#Обозначение необходимой информации: папки, список файлов, словари для хранения значений метрик
gan_folder = "GAN/"
ladi_folder = "LADI/"
prompt_folder = "PROMPT/"
source_folder = "image/"

ladi_files = [f for f in os.listdir(ladi_folder) if f.lower().endswith(".jpg")]
prompt_files = [f for f in os.listdir(prompt_folder) if f.lower().endswith(".jpg")]
source_files = [f for f in os.listdir(source_folder) if f.lower().endswith(".jpg")]
gan_files = [f for f in os.listdir(gan_folder) if f.lower().endswith(".png")]

#Подсчёт метрик для каждой модели
gan_scores = calculate_lpips(source_folder, gan_folder, source_files, gan_files)
ladi_scores = calculate_lpips(source_folder, ladi_folder, source_files, ladi_files)
prompt_scores = calculate_lpips(source_folder, prompt_folder, source_files, prompt_files)

#Вывод среднего значения для каждой модели
mean_score = [np.mean(gan_scores), np.mean(ladi_scores), np.mean(prompt_scores)]
models = ['pasta-gan++', 'ladi-viton', 'promptdresser']

for model, score in zip(models, mean_score):
    print(f"Средний LPIPS для {model}: {score:.4f}")

#Настройка графика (BarPlot)
plt.figure(figsize=(10, 8))
bars = plt.bar(models, mean_score, color=['skyblue', 'lightgreen', 'tomato'])
plt.ylim(0, max(mean_score)*1.1)

plt.ylabel('Средний LPIPS')
plt.title('Результаты подсчёта LPIPS для моделей')
plt.grid(axis='y', alpha=.6)

#Добавление значений на график
for bar in bars:
    yval = bar.get_height()
    plt.text(bar.get_x() + bar.get_width()/2, yval + 0.01, f"{yval:.4f}", ha='center', va='bottom')

#Отображение графика, его сохранение
plt.show()
plt.savefig("lpips_results.png")
\end{minted}