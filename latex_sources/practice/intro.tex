\section{Практическая часть}
\subsection{Инструменты}
В рамках практической части курсовой работы анализируются популярные и эффективные решения задачи виртуальной примерочной в компьютерном зрении. Эти решения, в свою очередь, используют определённые подходы и библиотеки, которые обеспечивают эффективную и более простую предобработку данных (в частности изображений), а также обучение и использование предобученных моделей глубокого обучения. Далее рассматриваются наиболее популярные библиотеки и инструменты, которые применяются в проектах для решения задачи генеративного компьютерного зрения.

Все рассматриваемые проекты написаны на языке \texttt{Python}, поскольку он является одним из самых популярных в задачах машинного обучения и искусственного интеллекта. Популярность данного языка в сфере машинного обучения связана с простотой синтаксиса, а также с большим количеством специализированных библиотек для реализации решений в этой сфере. 

\subsubsection{Библиотеки для работы с изображениями}
Для ускорения операций с многомерными массивами и числовыми данными в нём активно используется библиотека \texttt{NumPy} (\texttt{Numerical Python}), которая предоставляет ускоренную работу с данными, а также функции для математических вычислений. Это ускорение позволяет более эффективно предобрабатывать входные данные, исполнять большие вычисления и аггрегировать данные \cite{numpy}.

Для использования модели, зачастую, изображения необходимо изначально отредактировать. Для этих целей используется библиотека \texttt{OpenCV} (\texttt{python-opencv}). В силу большого количества полезных и эффективных в рамках компьютерного зрения функций, \texttt{OpenCV} отлично подходит для реализации загрузки, предобработки и редактирования изображений \cite{opencv}.

\subsubsection{Фреймворки для глубокого обучения}
Ключевым компонентом каждого из анализируемых проектов является использование фреймворка \texttt{PyTorch}, который обеспечивает удобные средства для создания и обучения глубоких нейронных сетей. В частности, модули \texttt{torch} и \texttt{torchvision} помогают реализовывать сложные генеративные модели, а также использовать готовые архитектуры и инструменты для аугментации и обработки изображений, что ускоряет процесс разработки и повышает качество моделей. Помимо этого, \texttt{PyTorch} позволяет написать собственную обработку набора данных, что делает доступным обучение моделей любого уровня сложности. \texttt{PyTorch} использует объектно"=ориентированный \texttt{API}, что подразумевает инициализацию моделей и обработчиков наборов данных с помощью классов и концепций ООП \cite{torch}.

Важной составляющей, необходимой для эффективного и быстрого обучения моделей, является использование технологии \texttt{CUDA}. \texttt{CUDA} "--- платформа для параллельных вычислений, предложенная компанией \texttt{NVIDIA}. Данная технология позволяет эффективнее использовать ресурсы видеокарты, а также сократить время, затрачиваемое на обучение моделей. В совокупности с вышеперечисленными библиотеками, \texttt{CUDA} позволяет создавать сложные и высокотребовательные модели, отлаживать и обучать их \cite{cuda}.

\subsubsection{Сериализация моделей}
Для сохранения конечной модели, а также для использования предобученных моделей применяется модуль \texttt{pickle}, который обеспечивает сериализацию объектов \texttt{Python} в удобный формат для последующего восстановления и повторного использования без необходимости повторной обработки данных. Это позволяет составлять более сложные архитектуры, не теряя времени на повторном обучении моделей.

